
\subsection{VDTA: Defining Safety Policies for \ac{CPS}}

We consider our industrial \ac{CPS} systems to have finite ordered sets of valued input channels ${I} = \{{i_1}, {i_2}, \ldots {i_n}\}$ and valued output channels ${O} = \{{o_1}, {o_2}, \ldots {o_n}\}$.

A VDTA can be seen as a DTA with a finite set of locations, a finite set of discrete clocks used to represent time evolution, and external input (resp. output channels) called ``external variables'' which are used for representing system data.
They also have internal variables which are used for internal computation, compared to the external variables which model the data carried by the actions from the monitored system (resp. environment). 
\ignore{
In a VDTA, a transition consists of an action carrying values of external variables, a guard on internal variables, external variables and clocks, and an assignment of internal variables, and reset of clocks.
%Thus, each input (resp. output channel) is considered as an external variable. 

\begin{remark}[Actions in VDTA]
In the framework that we consider, there is only a default ``tick'' action, which represents the ticking of a ``logical clock'' called ``tk''. 
We thus simplify the model by omitting actions.
Every transition consists of a default ``tick'' action. In every tick, the values of all input-output channels are updated.   
\end{remark}
}
Within \ac{VDTA}, there is an implicit logical tick similar to synchronous programming languages~\cite{SynchronousLanguages12YearsLater}.
All transitions occur relative to this tick.
At the start of a tick, all input channels are sampled, and at the end of the tick, all output channels are emitted.
Thus, a tick constitutes and atomic reaction of the reactive system.
Before we look into the formal definition of VDTA, let us consider an example.

\begin{figure}[tb]
	\centering
	\input{../fig/dta/prop_overcurrent_bidir_paper3b}
	\caption{Complete overcurrent policy defined via \ac{VDTA} $\mathcal{A}_{oc}$}
	\label{fig:vsa-overcurrent}
	\vspace{-5mm}
\end{figure}


\begin{example}
	\label{eg:vdta}
	%we need values because of tau
	%deadline is a function of the overcurrent value
	The overcurrent relay case study can be presented as a \ac{VDTA} $\mathcal{A}_{oc}$, as depicted in Figure~\ref{fig:vsa-overcurrent}. 
	This \ac{VDTA} formally specifies two properties.
	Firstly, entering a set current value that is beyond a safe maximum is a violation.
	Secondly, staying in an overcurrent state for too long without tripping the relay is a violation.
	
	The \ac{VDTA} for these rules has a set of locations $L = \{ l_{safe},l_{unsafe}, l_{vio}\}$, with accepting locations F = $\{l_{safe}\}$, and $l_{safe}$ as the initial state. 
	It has the set of external variables  $C = \{i, i_{set}, r\}$, where $i$ and $i_{set}$ (representing the measured and set currents) are input channels of integer type, and $r$ (representing the relay command) is an output channel of Boolean type.  
	Its set of internal variables $V = \{i_{max}\}$, where $i_{max}$ (representing the physical max current capacity of the system) is of integer type.
	
%	The \ac{VDTA} has a set of actions $\Sigma = \{tk(i, i_{set}, r)\}$.
%	It consists of only a default ``tick'' action, which represents the ticking of a ``logical clock'' called ``tk''. 
%	In every tick, the values of all input-output channels are updated.
	
	In a VDTA, a transition can have guards on internal variables, external variables and clocks. 
	For example, when in location $l_{safe}$, the transition from $l_{safe}$ to $l_{unsafe}$ can happen only if $r$ is true, and $i \geq i_{set}$ and $i_{set} \leq i_{max}$. 
	The clock values can be reset upon transitions. For example, upon transition from $l_{safe}$ to $l_{unsafe}$, the value of clock $x$ is reset to 0.
	The transitions can also depend on computable functions.
	For instance, $\tau$ is a function with $i$ and $i_{set}$ as input parameters, which returns a ``safe time'' value.
	
	It can be seen that there are three violation transitions, labelled \textcircled{a}, \textcircled{b}, and \textcircled{c}.
	Transitions \textcircled{a} and \textcircled{b} occur when ${i_{set}}$ is greater than the safe physical value $i_{max}$, and represent the implementation of the first property.
	Transition \textcircled{c} encodes the second property, that is the case where the overcurrent time exceeds the safe time without the circuit-breaker `trip' command being issued.
\end{example}

Let us now consider in more detail the syntax and semantics of VDTAs.
%
For a variable (resp. channel) $v$, ${\cal D}_v$ denotes its domain,
and for a tuple of variables $V= (v_1, \ldots, v_n)$,
${\cal D}_V$ is the product domain ${\cal D}_{v_1} \times \cdots \times {\cal D}_{v_n}$.
%A predicate $P(V)$ on a tuple of variables $V$ is a logical formula whose semantics is a function ${\cal D}_V \rightarrow \{\true, \false\}$.
A valuation of the variables in $V$
is a mapping $\nu$ which maps every variable $v \in V$ to a value $\nu(v)$ in ${\cal D}_v$.
%
Let $X=\{x_1,\ldots, x_k\}$ be a finite set of integer variables representing discrete clocks.
%
A {\em valuation} for $x$ is an element of $\bbn$, that is a function from $x$ to $\bbn$.
The set of valuations for the set of clocks $X$ is denoted by $\chi$.
%
For $\chi\in\bbn^X$, $\chi+1$ (which captures the ticking of the digital clock) is the valuation assigning $\chi(x)+1$ to each clock variable $x$ of $X$.
Given a set of clock variables $X' \subseteq X$, $\chi[X' \leftarrow 0]$ is the valuation of clock variables $\chi$ where all the clock variables in $X'$ are assigned to $0$.



%
\begin{definition}[Syntax of {VDTA}s]
	\label{def:ptav}
	A {VDTA} is a tuple \\
	$\calA = \left(L, {l_0}, X, V, C, \Theta, F,  \Delta \right)$ where:
	\squishlist
%	\item $\Sigma$ is a non-empty finite set of actions,
%	and an action $a \in \Sigma$ has a signature $sig(a) = ( t_0, t_1, \ldots, t_k )$ which is a tuple of types of the external variables,
	\item $L$ is a finite non-empty set of locations, with $l_0 \in L$ the initial location, and $F \subseteq L$ the set of accepting locations;
	\item $X$ is a finite set of discrete clocks;
	\item $V$ is a tuple of typed internal variables; 
	\item $C$ is a tuple of external variables, where $C = I \cup O$, where $I$ is the set of input channels, and $O$ is the set of output channels; 
	\item $\Theta\subseteq {\cal D}_{V }$ is an initial condition which is a computable predicate over $V$;
	\item $\Delta$ is a finite set of transitions, and each transition $t \in \Delta$ is a tuple $( l, G, A, l' )$
	also written\\
	$l \xrightarrow{G(V,C), V':=A(V,C)} l'$
	such that,
	\squishlist
	\item[\textbullet] $l, l' \in L$ are respectively the origin and target locations of the transition;
	%\item[\textbullet] $c$ is a tuple of external variables;
	\item[\textbullet] $G = G^D \wedge G^X$ is the guard where
	\squishlist
	\item[-] $G^D \subseteq {\cal D}_V \times {\cal D}_{C}$
	is a computable predicate over internal variables and external variables  in $V \cup C$;
	\item[-] $G^X$ is a clock constraint over $X$ defined as a Boolean combinations of constraints of the form $x \sharp f(V \cup C)$, where $x \in X$ and $f(V \cup C)$ is a computable function, and $\sharp \in \{ <, \leq, =, \geq, > \}$;
	\squishend
	\item[\textbullet] $A$$=$$(A^D, A^X)$ is the assignment of the transition where
	\squishlist
	\item[-] $A^D :{\cal D}_V \times {\cal D}_{C} \rightarrow {\cal D}_V$ defines the evolution of internal variables.
	\item[-] $A^X \subseteq X$ is the set of clocks to be reset.
	\squishend
	\squishend
	\squishend
\end{definition}
%

A word is a sequence $\sigma = \eta_1\cdot \eta_2 \cdots \eta_n$ where $\forall i \in [1,n]:$ $\eta_i$ is a tuple of values of variables in $C = I \cup O$. 


\ignore{
\begin{example}
	\label{eg:vdta}
	%we need values because of tau
	%deadline is a function of the overcurrent value
	Our case study can be presented as a \ac{VDTA} $\mathcal{A}_{oc}$, as depicted in Figure~\ref{fig:vsa-overcurrent}. 
	This \ac{VDTA} specifies that staying in an overcurrent state for too long without tripping the relay is a violation, and that entering a set current value that is beyond a safe maximum is a violation.
	We encode this as a \ac{VDTA} with set of locations $L = \{ l_{safe},l_{unsafe}, l_{vio}\}$, with accepting locations F = $\{l_{safe}\}$, and $l_{safe}$ is the initial state. The set of external variables  $C = \{i, i_{set}, r\}$, where $i$ and $i_{set}$ are input channels of type integer, and $r$ is an output channel of Boolean type.  
	The set of actions $\Sigma = \{tk(i, i_{set}, r)\}$, and the set of internal variables $V = \{i_{max}\}$, where $i_{max}$ is of type integer.
	In the example, $\tau$ is a function with $i$ and $i_{set}$ as input parameters. 
	
	%We use the I/O specified in Example~\ref{ex:io}. 
	
	
	
	It can be seen that there are three violation transitions, labelled \textcircled{a}, \textcircled{b}, and \textcircled{c}.
	Transition \textcircled{a} occurs when ${i_{set}}$ is greater than some safe physical value $i_{max}$, which is a physical property of the system itself. 
	Likewise, \textcircled{b} also checks this property, but from location $l_{unsafe}$ instead of $l_{safe}$.
	Finally, transition \textcircled{c} represents the case where the overcurrent time exceeds the safe time without the circuit breaker `trip' command being issued.
\end{example}
}

Policy \ac{VDTA} are required to be \textit{deterministic}, i.e. for any given state, the conjunction of any guards of any other outgoing transitions may not be satisfiable; and \textit{complete}, i.e. for any given state at any given time and any valuation of variables in $C$, at least one transition guard is satisfied.
%%%%%%%%%%%%%%%%%%%%%%%
%%%%%%%%%%%%%%%%%%%%%%%
\subsection{Semantics for \ac{VDTA}}

Let $\calA = \left(\Sigma, L, {l_0}, X, V, C, \Theta, F,  \Delta \right)$  be a VDTA.
The semantics of $\calA$ is a timed transition system,
where a state consists of a location, and valuations of internal variables $V$ and clocks $X$.
Each transition is associated with values of external variables in $C$.

\begin{definition}[Semantics of {VDTA}s]
	\label{def:vdta:semantics}
	The semantics of $\calA$ is a timed transition system $\sem{\calA}=( Q, q_0, Q_F, \Gamma, \to )$, defined as follows:
	\squishlist
	\item $Q = L \times {\cal D}_V \times \bbn^X$, is the set of states of the form $q= ( l,\nu ,\chi )$ where
	$l \in L$ is a location,
	$\nu \in {\cal D}_V$ is a valuation of internal variables,
	$\chi$ is a valuation of clocks;
	\item $Q_0 = \{ ( l_0,\nu, \chi_{[X \leftarrow 0]} )  \mid \Theta(\nu)=\true \}$ is the set of initial states;
	\item $Q_F = F \times {\cal D}_V \times \bbn^X$ is the set of accepting states;
	\item $\Gamma = \{ \eta \mid
	\eta \in {\cal D}_{C}  \}$ is the set of transition labels;
	\item $\to\subseteq Q\times \Gamma\times Q$  the transition relation
	is the smallest set of transitions of the form
	$( l,\nu,\chi \rangle \longrightarrow {\eta} \langle l',\nu',\chi')$
	such that  $\exists ( l, G, A, l' ) \in \Delta$,
	with $G^X(\chi + 1) \wedge G^D(\nu, \eta) $ evaluating to {\true},
	$\nu'= A^D(\nu, \eta)$ and $\chi'=(\chi+1)[A^X \leftarrow 0]$.
	\squishend
\end{definition}


%The set of timed words over $\Sigma$ where the actions carry parameter value and other data is denoted by $\tw(\Lambda)$.

A {\em run} $\rho$ of $\sem{\calA}$ from a state $q\in Q$ over a {\em trace} $w =  \eta_1\cdot \eta_2\cdots \eta_n$ is a sequence of moves in $\sem{\calA}$:
$\rho = q \xrightarrow {\eta_1} q_1
\cdots q_{n-1}\xrightarrow {\eta_n} q_{n}$,
for some $n\in\bbn$.
A run is accepted if it starts from the initial state $q_0\in Q$ and ends in an accepted state $q_n \in Q_F$.
%The set of runs from the initial state $q_0\in Q$,  is denoted $\Run(\calA)$ and $\Run_{Q_F}(\calA)$ denotes the subset of those runs {\em accepted} by $\calA$, i.e.,  ending in an accepting state $q_n \in Q_F$.

%%%%%%%%%%%%%%%%%%%%%%%%%%%%%%%%%%%%%%%%%%%%%%%
%%%%%%%%%%%%%%%%%%%%%%%%%%%%%%%%%%%%%%%%%%%%%%%
\begin{example}[Run of a VDTA]
%Let us consider the VDTA %discussed in Example~\ref{eg:vdta}
%presented in Figure~\ref{fig:vsa-overcurrent}. 
An example run of the VDTA depicted in Figure~\ref{fig:vsa-overcurrent} is elaborated here.
Let the internal variable $i_{max}$ be initialized with 10000.
A run of this VDTA starting from the initial state $(l_{safe}, i_{max}=10000, x = 0)$ for the word $\sigma = (4000, 5000,1)\cdot (8000, 5000,1)\cdot (7000, 5000,1)\cdot (8000, 5000,1)\cdot (8000, 5000,1)\cdot (8000, 5000,1)$ is:\\
{\small$
(l_{safe}, i_{max}=10000, x = 0)
\xrightarrow {(4000, 5000,1)} \\
(l_{safe}, i_{max}=10000, x = 1)
\xrightarrow {(8000, 5000,1)} \\
(l_{unsafe}, i_{max}=10000, x = 0)
\xrightarrow {(7000, 5000,1)} \\
(l_{unsafe}, i_{max}=10000, x = 1)
\xrightarrow {(8000, 5000,1)} \\
(l_{unsafe}, i_{max}=10000, x = 2)
\xrightarrow {(8000, 5000,1)} \\
(l_{unsafe}, i_{max}=10000, x = 3)
\xrightarrow {(8000, 5000,1)} \\
(l_{vio}, i_{max}=10000, x = 4).
$
}

The run started in the initial state and ended in a non-accepting state. It is thus a non-accepting run and represents a violation scenario.
\end{example}

