
\section{Related Work}
Functional safety, using standards such as IEC61508~\cite{iec61508}, ensures that the system meets reasonable level of safety by active mitigation of risks. 
Hardware safety is ensured by reliability analysis while software safety is ensured 
by the use of systematic development methods. There has been many attempts in the direction of functional safety of 
 \acp{ANN}~\cite{functional-safety, scann, SCANNStandard, EstSafeCriteria2003, ANNDevModel1999, ANN-test}. 
 \ignore{Removed reference ANNSafetyLifecycle2003}
IEC61508, from which many other standards are 
derived, uses safety integrity levels (SIL) and higher SIL levels recommend the use of formal methods.

As neural networks are widely used in safety critical systems, there is significant interest in formal methods for AI. Tripakis discusses
the need for convergence in methods between model-driven design and data-driven design in~\cite{tripakis2018data}. 
Okano et. al. explore model checking using timed automata~\cite{timed-enf-autonomous}. Also, the use of \acf{SMT} is explored recently for the 
robustness verification of \acp{DNN}\cite{Gehr2018AI2SA,reluplex}.
\ignore{Removed reference DeepANNverify}
However, static verification still poses considerable challenges~\cite{seshia2016towards}.

\ac{CPS} require both functional and timing verification. The issue of timing verification has received scant attention except 
the recent proposal for \ac{WCET} analysable neural networks~\cite{sann}. No proposal exists to date for automatic verification 
of both functionality and timing. Considering this, we  propose \acp{ENN} by combining neural networks with \acf{RE}. 

While \ac{RE} for transformational systems~\cite{theoryRE} is 
a well developed area, \ac{RE} for reactive systems is an emerging area of research. Initial proposals were uni-directional enforcers~\cite{bloem2015shield}. 
More recently, bi-directional enforcement is developed for CPS~\cite{theoryRE}.
However, these operate over Boolean streams and hence not general enough for \acp{ANN}. 
Another recent work \cite{Leucker:2018} proposed an approach for runtime monitoring of real-time valued signals. 
However, the approach only deals with runtime verification (checking for satisfaction of a given property at runtime).
This paper generalises the problem of bi-directional enforcement over 
valued input and output including timed properties for the first time.


\ignore{
The concept of \ac{RE} of autonomous systems has received some research attention. 
De Niz et. al. propose a type of \ac{RE} they term temporal enforcement, which ensures that the system controller meets timing deadlines where outputs are concerned~\cite{safe-enforce-auto}. 
While this shares similarities with the work in this paper, their work does not expand to cover \acp{ANN}, and does not propose the use of \ac{RE} for anything other than meeting timing deadlines.
Aniculaesei et. al. propose static formal verification and output run-time monitoring of basic, semi-autonomous, robotic systems to prevent physical collisions during system execution~\cite{runtime-monitor}.
The controller used in their system does not contain form of \ac{AI} and their enforcer only monitors the controllers outputs, unlike our approach which uses previously introduced \acp{SNN} as \ac{AI} controllers and supports bi-directional run-time enforcement.

Roop et al~\cite{sann} introduced \acfp{SNN}.
\acf{WCET} timing verification of these \acp{SNN} was possible on time predictable processors.
As such, these \acp{SNN} could be used in systems with strict timing deadlines.
\acp{SNN}, however, do not provide any \emph{functional} verification or safety to these systems.
}
